\documentclass[11pt,a4paper]{article}
\usepackage[utf8]{inputenc}
\usepackage[T1]{fontenc}
\usepackage{amsmath,amssymb}
\usepackage{graphicx}
\usepackage{xcolor}
\usepackage{listings}
\usepackage{hyperref}
\usepackage{geometry}
\usepackage{booktabs}
\usepackage{fancyhdr}
\usepackage{tcolorbox}
\usepackage{algorithm}
\usepackage{algpseudocode}

\geometry{margin=1in}
\pagestyle{fancy}
\fancyhf{}
\rhead{\thepage}
\lhead{CoSMoS User Manual}

\hypersetup{
    colorlinks=true,
    linkcolor=blue,
    filecolor=magenta,      
    urlcolor=cyan,
    citecolor=green,
    pdftitle={CoSMoS User Manual},
    pdfauthor={CoSMoS Development Team},
}

\lstdefinestyle{jsonStyle}{
    basicstyle=\ttfamily\small,
    breaklines=true,
    backgroundcolor=\color{gray!10},
    frame=single,
    numbers=left,
    numberstyle=\tiny\color{gray},
    keywordstyle=\color{blue},
    stringstyle=\color{red},
    commentstyle=\color{green!50!black},
    morestring=[b]",
}

\lstdefinestyle{bashStyle}{
    basicstyle=\ttfamily\small,
    breaklines=true,
    backgroundcolor=\color{blue!5},
    frame=single,
    language=bash,
}

\title{\textbf{CoSMoS User Manual}\\
\large Global Structure Search Program\\
Version 1.0}
\author{lipai@mail.sim.ac.cn}
\date{\today}

\begin{document}

\maketitle
\tableofcontents
\newpage

%==============================================================================
\section{Introduction}
%==============================================================================

\subsection{Overview}
CoSMoS (Global Structure Search Program) is a sophisticated computational tool designed to find stable atomic structures through advanced global optimization algorithms. It combines Monte Carlo sampling with a climbing algorithm to efficiently explore the potential energy surface and identify low-energy structural configurations.

\subsection{Key Features}
\begin{itemize}
    \item \textbf{Multi-calculator support}: Compatible with various energy calculators including EAM, CHGNet, DeepMD, FAIRChem, VASP, LAMMPS, and custom Python calculators
    \item \textbf{Flexible mobility control}: Supports index-based and region-based constraints on atomic movement
    \item \textbf{Permutation-invariant structure comparison}: Uses sorted SOAP descriptors combined with energy gating for robust duplicate detection
    \item \textbf{Automatic geometry classification}: Detects cluster, wire, slab, or bulk structures with appropriate handling of vacuum axes
    \item \textbf{Energy-weighted sampling}: Prioritizes movement of unstable atoms through per-atom energy analysis
\end{itemize}

%==============================================================================
\section{Theoretical Background}
%==============================================================================

\subsection{Stochastic Surface Walking (SSW) Algorithm}
CoSMoS is based on the Stochastic Surface Walking method \cite{shang2013ssw}, which combines:

\begin{enumerate}
    \item \textbf{Random displacement}: Structure perturbation along a weighted random direction
    \item \textbf{Climbing phase}: Addition of Gaussian bias potentials to escape local minima
    \item \textbf{Local optimization}: Minimization to nearest stationary point
    \item \textbf{Metropolis acceptance}: Temperature-dependent Monte Carlo acceptance
\end{enumerate}

\subsection{Algorithm Workflow}

\begin{algorithm}
\caption{CoSMoS Search Algorithm}
\begin{algorithmic}[1]
\State Initialize structure $\mathbf{R}_0$ and energy pool $\mathcal{E} = \emptyset$
\State Optimize $\mathbf{R}_0 \rightarrow \mathbf{R}_{\text{min}}$, add to pool
\For{$\text{step} = 1$ to $N_{\text{MC}}$}
    \State Generate random direction $\mathbf{N}$ (energy-weighted)
    \State Displace: $\mathbf{R}' = \mathbf{R}_{\text{current}} + d_s \mathbf{N}$
    \For{$h = 1$ to $H$ (Gaussian potentials)}
        \State Add Gaussian bias: $V_{\text{bias}} = \sum_{i=1}^{h} w e^{-|\mathbf{R} - \mathbf{R}_i|^2 / (2\sigma^2)}$
        \State Optimize on biased PES: $\mathbf{R}_h \rightarrow \mathbf{R}_{\text{climb}}$
        \If{converged or max steps reached}
            \State \textbf{break}
        \EndIf
    \EndFor
    \State Optimize on real PES: $\mathbf{R}_{\text{climb}} \rightarrow \mathbf{R}_{\text{new}}$
    \State Compute $E_{\text{new}} = E(\mathbf{R}_{\text{new}})$
    \If{$\mathbf{R}_{\text{new}}$ is not duplicate}
        \State Add $(\mathbf{R}_{\text{new}}, E_{\text{new}})$ to pool
    \EndIf
    \State Accept/reject via Metropolis: $P = \min(1, e^{-\Delta E / k_B T})$
    \State Update $\mathbf{R}_{\text{current}}$
\EndFor
\State \Return lowest energy structure from pool
\end{algorithmic}
\end{algorithm}

\subsection{Energy-Weighted Direction Sampling}
Unlike traditional Maxwell-Boltzmann sampling, CoSMoS uses per-atom potential energies to guide sampling:

\begin{equation}
\mathbf{N}_i \sim \mathcal{N}(0, \sigma_i^2), \quad \sigma_i = \sqrt{k_B T} \cdot e^{E_i^{\text{norm}}}
\end{equation}

where $E_i^{\text{norm}}$ is the normalized energy of atom $i$ relative to the lowest-energy atom of the same element. This prioritizes movement of high-energy, unstable atoms.

\subsection{Structure Comparison via SOAP Descriptors}
To handle atomic permutations, CoSMoS uses permutation-invariant descriptors:

\begin{enumerate}
    \item Compute per-atom SOAP vectors $\{\mathbf{s}_i\}_{i=1}^N$
    \item Sort by L2 norm: $\mathbf{s}_{\sigma(1)}, \ldots, \mathbf{s}_{\sigma(N)}$ where $\|\mathbf{s}_{\sigma(i)}\| \leq \|\mathbf{s}_{\sigma(i+1)}\|$
    \item Flatten to structure descriptor: $\mathbf{D} = [\mathbf{s}_{\sigma(1)}; \ldots; \mathbf{s}_{\sigma(N)}]$
    \item Compare: structures are duplicates if $\|\mathbf{D}_1 - \mathbf{D}_2\| < \tau_{\text{desc}}$ and $|E_1 - E_2| < \tau_E$
\end{enumerate}

%==============================================================================
\section{Installation}
%==============================================================================

\subsection{System Requirements}
\begin{itemize}
    \item Python 3.8 or higher
    \item pip (Python package manager)
    \item ASE $\geq$ 3.26.0 (Atomic Simulation Environment)
    \item dscribe $\geq$ 2.1.0 (for SOAP descriptors)
\end{itemize}

\subsection{Installation Steps}

\begin{lstlisting}[style=bashStyle, caption={Installation from source}]
# Clone the repository
git clone https://github.com/lipai-ustc/CosMos.git
cd cosmos

# (Optional) Create Conda environment
conda create -n cosmos python=3.10 -y
conda activate cosmos

# Install CoSMoS
pip install .

# (Optional) Install additional calculators
pip install deepmd-kit fairchem-core
\end{lstlisting}

\begin{tcolorbox}[colback=yellow!10,colframe=orange!50,title=Development Mode]
For active development, use editable installation:
\begin{lstlisting}[style=bashStyle]
pip install -e .
\end{lstlisting}
This allows code modifications without reinstallation.
\end{tcolorbox}

%==============================================================================
\section{Basic Usage}
%==============================================================================

\subsection{Input Files}
CoSMoS requires two input files in the working directory:

\begin{enumerate}
    \item \texttt{input.json}: Calculation parameters configuration
    \item \texttt{init.xyz}: Initial atomic structure in XYZ format
\end{enumerate}

\subsection{Running CoSMoS}
\begin{lstlisting}[style=bashStyle, caption={Basic execution}]
# Navigate to directory containing input.json and init.xyz
cd /path/to/working/directory

# Run search
cosmos
\end{lstlisting}

\subsection{Output Files}
Results are saved in the \texttt{cosmos\_output/} directory:

\begin{itemize}
    \item \texttt{all\_minima.xyz}: All discovered minimum energy structures (trajectory format)
    \item \texttt{best\_str.xyz}: Lowest energy structure
    \item \texttt{cosmos\_log.txt}: Detailed search log with energies and algorithm status
\end{itemize}

%==============================================================================
\section{Configuration Reference}
%==============================================================================

\subsection{input.json Structure}
The configuration file is divided into several logical sections:

\begin{lstlisting}[style=jsonStyle, caption={Minimal input.json example}]
{
  "system": {
    "name": "MySystem"
  },
  "potential": {
    "type": "eam",
    "model": "potential.eam.alloy"
  },
  "monte_carlo": {
    "steps": 100,
    "temperature": 300
  },
  "climbing": {
    "gaussian_height": 0.1,
    "gaussian_width": 0.2,
    "max_gaussians": 14,
    "optimizer": {
      "max_steps": 500,
      "fmax": 0.05
    }
  },
  "output": {
    "directory": "cosmos_output"
  }
}
\end{lstlisting}

\subsection{Potential Configuration}

\subsubsection{EAM Potential}
\begin{lstlisting}[style=jsonStyle]
"potential": {
  "type": "eam",
  "model": "AlCu.eam.alloy"
}
\end{lstlisting}

\subsubsection{CHGNet}
\begin{lstlisting}[style=jsonStyle]
"potential": {
  "type": "chgnet",
  "model": "pretrained"
}
\end{lstlisting}

\subsubsection{DeepMD}
\begin{lstlisting}[style=jsonStyle]
"potential": {
  "type": "deepmd",
  "model": "dp_model.pb"
}
\end{lstlisting}

\subsubsection{FAIRChem (Open Catalyst Project)}
\begin{lstlisting}[style=jsonStyle]
"potential": {
  "type": "fairchem",
  "model": "EquiformerV2-31M-S2EF-OC20-All+MD",
  "device": "cuda",
  "task_name": "oc20"
}
\end{lstlisting}

\subsubsection{VASP}
\begin{lstlisting}[style=jsonStyle]
"potential": {
  "type": "vasp",
  "model": "INCAR"
}
\end{lstlisting}
Reads VASP parameters from the specified INCAR file.

\subsubsection{Custom Python Calculator}
\begin{lstlisting}[style=jsonStyle]
"potential": {
  "type": "python"
}
\end{lstlisting}
Requires a \texttt{calculator.py} file in the working directory defining a \texttt{calculator} variable with an ASE Calculator object.

\subsection{Monte Carlo Parameters}
\begin{table}[h]
\centering
\begin{tabular}{lll}
\toprule
\textbf{Parameter} & \textbf{Description} & \textbf{Default} \\
\midrule
\texttt{steps} & Total MC steps & 100 \\
\texttt{temperature} & Temperature (K) for Metropolis & 300 \\
\bottomrule
\end{tabular}
\caption{Monte Carlo configuration parameters}
\end{table}

\subsection{Climbing Phase Parameters}
\begin{table}[h]
\centering
\begin{tabular}{lll}
\toprule
\textbf{Parameter} & \textbf{Description} & \textbf{Default} \\
\midrule
\texttt{gaussian\_height} & Gaussian bias height $w$ (eV) & 0.1 \\
\texttt{gaussian\_width} & Step size $d_s$ (\AA) & 0.2 \\
\texttt{max\_gaussians} & Max Gaussians per climb $H$ & 14 \\
\texttt{optimizer.max\_steps} & Optimization max steps & 500 \\
\texttt{optimizer.fmax} & Force convergence (eV/\AA) & 0.05 \\
\bottomrule
\end{tabular}
\caption{Climbing phase configuration parameters}
\end{table}

\subsection{Mobility Control}

Mobility control allows constraining which atoms can move during optimization. Three modes are supported:

\subsubsection{Mode 1: All Atoms Mobile (Default)}
\begin{lstlisting}[style=jsonStyle]
"mobility_control": null
\end{lstlisting}
or omit the \texttt{mobility\_control} section entirely.

\subsubsection{Mode 2a: Specify Mobile Atoms}
\begin{lstlisting}[style=jsonStyle]
"mobility_control": {
  "mode": "indices_free",
  "indices_free": [10, 20, 25],
  "wall_strength": 10.0,
  "wall_offset": 2.0
}
\end{lstlisting}
Only listed atoms (0-based indexing) are mobile; all others are fixed.

\subsubsection{Mode 2b: Specify Fixed Atoms}
\begin{lstlisting}[style=jsonStyle]
"mobility_control": {
  "mode": "indices_fix",
  "indices_fix": [0, 1, 2],
  "wall_strength": 10.0,
  "wall_offset": 2.0
}
\end{lstlisting}
Listed atoms are fixed; all others are mobile.

\subsubsection{Mode 3a: Spherical Region}
\begin{lstlisting}[style=jsonStyle]
"mobility_control": {
  "mode": "region",
  "region_type": "sphere",
  "center": [5.0, 5.0, 5.0],
  "radius": 10.0,
  "wall_strength": 10.0,
  "wall_offset": 2.0
}
\end{lstlisting}
Atoms within the sphere are mobile.

\subsubsection{Mode 3b: Slab Region}
\begin{lstlisting}[style=jsonStyle]
"mobility_control": {
  "mode": "region",
  "region_type": "slab",
  "origin": [0.0, 0.0, 0.0],
  "normal": [0, 0, 1],
  "min_dist": -5.0,
  "max_dist": 5.0,
  "wall_strength": 10.0,
  "wall_offset": 2.0
}
\end{lstlisting}
Atoms between two parallel planes are mobile.

\subsubsection{Wall Potential}
When \texttt{wall\_strength} $> 0$, a quadratic repulsive potential prevents mobile atoms from penetrating into immobile regions:

\begin{equation}
V_{\text{wall}} = \begin{cases}
0 & \text{if } d \leq d_{\text{offset}} \\
\frac{1}{2} k_{\text{wall}} (d - d_{\text{offset}})^2 & \text{if } d > d_{\text{offset}}
\end{cases}
\end{equation}

where $d$ is the penetration distance, $d_{\text{offset}}$ is \texttt{wall\_offset}, and $k_{\text{wall}}$ is \texttt{wall\_strength}.

%==============================================================================
\section{Low-Dimensional Systems}
%==============================================================================

\subsection{Geometry Classification}
CoSMoS automatically classifies structures based on vacuum analysis:
\begin{itemize}
    \item \textbf{Cluster (0D)}: Vacuum along all three axes
    \item \textbf{Wire (1D)}: Vacuum along two axes
    \item \textbf{Slab (2D)}: Vacuum along one axis
    \item \textbf{Bulk (3D)}: No significant vacuum
\end{itemize}

\subsection{Structure Preparation}
\begin{tcolorbox}[colback=red!5,colframe=red!50,title=IMPORTANT: Structure Centering]
For low-dimensional systems, you \textbf{must manually center} the structure in the simulation box before running CoSMoS. The geometry detection algorithm measures distances from atoms to cell boundaries; misplaced structures will be misclassified.

\textbf{Recommended placement:}
\begin{itemize}
    \item \textbf{0D cluster}: Center at $(L_x/2, L_y/2, L_z/2)$
    \item \textbf{1D wire}: Wire extends along one axis, centered in the other two
    \item \textbf{2D slab}: Centered along the vacuum direction
\end{itemize}
\end{tcolorbox}

\subsection{Vacuum Axis Handling}
For structures with vacuum axes, CoSMoS automatically:
\begin{enumerate}
    \item Detects vacuum axes using absolute coordinate analysis
    \item Centers the structure along vacuum axes to the box center
    \item Applies recentering after each accepted Monte Carlo move
    \item Prevents translational drift during search
\end{enumerate}

%==============================================================================
\section{Examples}
%==============================================================================

\subsection{Example 1: AlCu Cluster with EAM}
\begin{lstlisting}[style=bashStyle]
cd examples/AlCu-EAM
python generate_structure.py  # Generate init.xyz
cosmos                         # Run search
\end{lstlisting}

Configuration highlights:
\begin{itemize}
    \item EAM potential for Al-Cu system
    \item 100 Monte Carlo steps
    \item Temperature: 300 K
\end{itemize}

\subsection{Example 2: C$_{60}$ with DeepMD}
\begin{lstlisting}[style=bashStyle]
cd examples/C60-deepmd
# Ensure dp_model.pb is present
cosmos
\end{lstlisting}

Configuration highlights:
\begin{itemize}
    \item DeepMD neural network potential
    \item Requires \texttt{dp\_model.pb} trained model
    \item Install: \texttt{pip install deepmd-kit}
\end{itemize}

\subsection{Example 3: Au Surface with FAIRChem (Custom Calculator)}
\begin{lstlisting}[style=bashStyle]
cd examples/Au100-python-fairchem
# Check calculator.py for calculator definition
cosmos
\end{lstlisting}

Configuration highlights:
\begin{itemize}
    \item Custom Python calculator using FAIRChem
    \item Slab region mobility control for surface atoms
    \item Requires: \texttt{pip install fairchem-core}
\end{itemize}

%==============================================================================
\section{Advanced Topics}
%==============================================================================

\subsection{Custom Calculator Development}
To use a custom ASE-compatible calculator:

\begin{enumerate}
    \item Create \texttt{calculator.py} in your working directory
    \item Define a \texttt{calculator} variable
    \item Set \texttt{"type": "python"} in \texttt{input.json}
\end{enumerate}

Example \texttt{calculator.py}:
\begin{lstlisting}[language=Python, style=jsonStyle]
from ase.calculators.emt import EMT

# Define your calculator
calculator = EMT()
\end{lstlisting}

\subsection{Parallel Execution}
CoSMoS is designed for serial execution per search. For parallel exploration:
\begin{itemize}
    \item Run multiple independent searches with different initial structures
    \item Use different random seeds (implicitly different due to system time)
    \item Combine results post-processing
\end{itemize}

\subsection{Troubleshooting}

\subsubsection{Structure Misclassification}
\textbf{Symptom:} Cluster treated as bulk or vice versa.

\textbf{Solution:} Ensure structure is centered in the simulation box. Check with:
\begin{lstlisting}[language=Python, style=jsonStyle]
from ase.io import read
atoms = read('init.xyz')
pos = atoms.get_positions()
center = pos.mean(axis=0)
cell = atoms.get_cell()
box_center = (cell[0] + cell[1] + cell[2]) / 2
print(f"Structure center: {center}")
print(f"Box center: {box_center}")
\end{lstlisting}

\subsubsection{Slow Convergence}
\textbf{Possible causes:}
\begin{itemize}
    \item \texttt{gaussian\_height} too small: increase to 0.2--0.5 eV
    \item \texttt{temperature} too low: try 500--1000 K
    \item \texttt{max\_gaussians} too small: increase to 20--30
\end{itemize}

\subsubsection{Too Many Duplicates}
\textbf{Possible causes:}
\begin{itemize}
    \item \texttt{duplicate\_tol} too loose: default is 0.01, try 0.005
    \item Energy tolerance too large: adjust in source (default 1e-3 eV)
\end{itemize}

%==============================================================================
\section{Parameter Tuning Guidelines}
%==============================================================================

\begin{table}[h]
\centering
\begin{tabular}{lll}
\toprule
\textbf{Parameter} & \textbf{Effect} & \textbf{Tuning Strategy} \\
\midrule
\texttt{gaussian\_height} & Barrier crossing ability & Increase for rugged PES \\
\texttt{gaussian\_width} & Step size & Smaller for precise, larger for exploration \\
\texttt{max\_gaussians} & Climbing persistence & Increase if stuck in shallow minima \\
\texttt{temperature} & Acceptance probability & Higher for aggressive search \\
\texttt{monte\_carlo.steps} & Total exploration & More steps = better coverage \\
\bottomrule
\end{tabular}
\caption{Parameter tuning guidelines}
\end{table}

%==============================================================================
\section{References}
%==============================================================================

\begin{thebibliography}{9}

\bibitem{shang2013ssw}
Shang, C., \& Liu, Z.-P. (2013).
\textit{Stochastic Surface Walking Method for Structure Prediction and Pathway Searching}.
Journal of Chemical Theory and Computation, 9(5), 1838--1845.

\bibitem{shang2013jcp}
Shang, C., \& Liu, Z.-P. (2013).
\textit{Stochastic surface walking method for global optimization of atomic clusters and biomolecules}.
The Journal of Chemical Physics, 139(24), 244104.

\bibitem{jctc2012}
Zhang, X.-J., \& Shang, C., \& Liu, Z.-P. (2012).
\textit{Double-Ended Surface Walking Method for Pathway Building and Transition State Location of Complex Reactions}.
Journal of Chemical Theory and Computation, 9(12), 5745--5753.

\end{thebibliography}

%==============================================================================
\section{License}
%==============================================================================

This project is licensed under the GNU General Public License v3.0 (GPL-3.0).

You may copy, distribute, and modify this software under the terms of the GPL-3.0.
For the full license text, see: \url{https://www.gnu.org/licenses/gpl-3.0.html}

%==============================================================================
\appendix
\section{Complete Configuration Example}
%==============================================================================

\begin{lstlisting}[style=jsonStyle, caption={Full input.json with all options}]
{
  "system": {
    "name": "CompleteExample"
  },
  "potential": {
    "type": "deepmd",
    "model": "dp_model.pb"
  },
  "monte_carlo": {
    "steps": 200,
    "temperature": 500
  },
  "climbing": {
    "gaussian_height": 0.2,
    "gaussian_width": 0.25,
    "max_gaussians": 20,
    "optimizer": {
      "max_steps": 500,
      "fmax": 0.03
    }
  },
  "mobility_control": {
    "mode": "region",
    "region_type": "sphere",
    "center": [10.0, 10.0, 10.0],
    "radius": 8.0,
    "wall_strength": 15.0,
    "wall_offset": 1.5
  },
  "output": {
    "directory": "my_output"
  }
}
\end{lstlisting}

\end{document}
